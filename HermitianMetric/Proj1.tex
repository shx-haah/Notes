%!TEX program = Xelatex
\documentclass{ctexart}

\usepackage{amsmath, amssymb, amsthm, titlesec}
\usepackage[skins, breakable, theorems]{tcolorbox}
\usepackage{xparse, ocgx2, hyperref}
\usepackage[top=2.54cm, bottom=2.54cm, left=3.18cm, right=3.18cm]{geometry}

\def\Id{\,\mathrm{d}} % 积分中的正体d
\newcommand{\norm}[1]{\left\|#1\right\|} % 范数

% [number within=section/...]{}{<display name>}{<style>}{<lebal>(cite as "Thm:...")}
\newtcbtheorem[]{tcbdefinition}{Definition}{fonttitle = \bfseries}{Def}
\newtcbtheorem[]{tcbtheorem}{Theorem}{fonttitle = \bfseries}{Thm}
\newtcbtheorem[]{tcbproposition}{Proposition}{fonttitle = \bfseries}{Pro}
\newtcbtheorem[]{tcblemma}{Lemma}{fonttitle = \bfseries}{Lem}
\newtcbtheorem[]{tcbcorollary}{Corollary}{fonttitle = \bfseries}{Cor}
\NewDocumentEnvironment{definition}{ O{} O{} } % two optional arguments
  {\tcbdefinition{#1}{#2}}
  {\endtcbdefinition}
\NewDocumentEnvironment{theorem}{ O{} O{} }
  {\tcbtheorem{#1}{#2}}
  {\endtcbtheorem}
\NewDocumentEnvironment{proposition}{ O{} O{} }
  {\tcbproposition{#1}{#2}}
  {\endtcbproposition}
\NewDocumentEnvironment{lemma}{ O{} O{} }
  {\tcblemma{#1}{#2}}
  {\endtcblemma}
\NewDocumentEnvironment{corollary}{ O{} O{} }
  {\tcbcorollary{#1}{#2}}
  {\endtcbcorollary}
\makeatletter
\newcommand\tcb@cnt@tcbdefinitionautorefname{Definition}
\newcommand\tcb@cnt@tcbtheoremautorefname{Theorem}
\newcommand\tcb@cnt@tcbpropositionautorefname{Proposition}
\newcommand\tcb@cnt@tcblemmaautorefname{Lemma}
\newcommand\tcb@cnt@tcbcorollaryautorefname{Collory} 
\makeatother

\title{Note 1: Hermitian metric}
\date{} % 显示日期

\renewcommand\refname{Reference}

%%%%%%%%%%%%%%%%%%%%%%%%%%%%%%%%%%%%%%%%%%%%%%%%%%%%%
\begin{document}

\maketitle

In this note, we introuduce the definition of Hermitian structure together with some related konwledge, and most are about linear algebra.
Reference of this note is \cite{Huybrechts2005} and \cite{Kobayashi1996}. 

Let $X$ be a complex manifold. We denote the induced almost complex structure by $I$. 
\begin{definition}[Hermitian structure]
    A Riemannian metric $g$ on $X$ is an Hermitian structure on $X$ if for any point $x \in X$ the scalar product $g_{x}$ on $T_{x} X$ is compatible with the almost complex structure $I_{x}$. 
    The induced real $(1,1)$-form $\omega:=g(I(\ ),(\ ))$ is called the fundamental form.
\end{definition}

To figure out what the above means, we shall explain each term in detail. 

\subsection*{The almost complex structure on vector spaces}

First, we focus on the linear algebraic part, discussing the almost complex structure and the corresponding 
fundamental form on vector spaces. 

Let $V$ be a finite-dimensional real vector space.
\begin{definition}[Almost complex structure]
    An endomorphism $I: V \rightarrow V$ with $I^{2}=-$ id is called an almost complex structure on $V$.
\end{definition}
Clearly, if $I$ is an almost complex structure then $I \in \operatorname{GL}(V)$. 

If V is the real vector space underlying a complex vector space (a complex vector space but regarded as real vector space, and denoted by $(V,i)$ when we mention it as a complex vector space) 
then $v \mapsto i \cdot v$ defines an almost complex structure $I$ on $V$ (here $V$ refers to the real vector space). The converse holds true as well:
\begin{lemma}
    Let $V$ be real $n$-dimensional vector space and an almost complex structure $I$ is defined as above. Then,

    1. The dimension $n$ is an even integer.

    2. Via $I$, the real vector space $V$ can be made into a complex vector space such that the multiplication by complex numbers extends the multiplication by reals.

    3. With respect to the complex vector space structure on $V$ obtained in 2, $I: V \rightarrow V$ is a complex linear map.
\end{lemma}

\begin{proof}[Proof]
    1. Since $I^{2}(v)=-v,\  \forall v \in V ,\  I$ does not have any real eigenvalues (if $I(v)=\lambda v, v \neq 0 \Rightarrow I^{2}(v)=\lambda^{2} v=-v \Rightarrow \lambda^{2}=-1$).
    If $n$ is odd the characteristic polynomial of $I$ has odd degree therefore at least one real eigenvalue.

    Or easily, one can consider the determination of the equation $I^2=-1$.

    2. The $\mathbb{C}$-module structure (a vector space on field $\mathbb{P}$ is a $\mathbb{P}$-module) on $V$ is defined by $(a+i b) \cdot v=a \cdot v+b \cdot I(v)$, where $a, b \in \mathbb{R}$. 
    It is enough to verify: 
    \begin{itemize}
        \item $1\cdot v=v,\ \forall v\in V$, 
        \item $(\alpha_1+\alpha_2)v=\alpha_1 v+\alpha_2 v,\ \forall \alpha_1,\alpha_2\in \mathbb{C},\ v\in V$, 
        \item $\alpha(v_1+v_2)=\alpha v_1+ \alpha v_2,\ \forall \alpha\in \mathbb{C},\ v_1,v_2\in V$, 
        \item $(\alpha_1 \alpha_2)v=\alpha_1(\alpha_2v),\ \forall \alpha_1,\alpha_2\in \mathbb{C},\ v\in V$. 
    \end{itemize}
    By the $\mathbb{R}$-linearity of $I$, one can easily knows the first three. Together with the assumption $I^{2}=-$ id yield $((a+i b)(c+i d)) \cdot v=(a+i b)((c+i d) \cdot v)$ and 
    in particular $i(i \cdot v)=-v$.

    3. Verify that $I(\alpha v)=\alpha I(v),\ \forall \alpha\in \mathbb{C},\ v\in V.$
\end{proof}

Thus, almost complex structures and complex structures are equivalent notions for vector spaces, and the complex dimension of V is $\frac{n}{2}$ ($v$ and $I(v)$ are not $\mathbb{R}$-linearly dependent but
$\mathbb{C}$-linearly dependent). From now on, we will always assume the real dimension of $V$ is $2n$. 
\begin{corollary}
  Any almost complex structure on $V$ induces a natural orientation on $V$. 
\end{corollary}
\begin{proof}[Proof]
  Using the lemma, the assertion reduces to the statement that the real vector space $\mathbb{C}^n$ admits a natural orientation. 
  We may assume $n = 1$ and use the orientation given by the basis $(1, i)$. 
\end{proof}

Here we are about to discuss the almost complex structure on the complexification of a real vector space $V$, which will help us in the manifold situation. 

One can define the complexification of $V$ by taking the tensor product of $V$ with the complex numbers (thought of as a 2-dimensional vector space over the reals):
$$
V_{\mathbb{C}}=V \otimes_{\mathbb{R}} \mathbb{C}
$$
The subscript $\mathbb{R}$ on the tensor product indicates that the tensor product is taken over the real numbers (since $V$ is a real vector space this is the only sensible option anyway, 
so the subscript can safely be omitted). As it stands, $V_{\mathbb{C}}$ is only a real vector space. However, we can make $V_{\mathbb{C}}$ into a complex vector space 
by defining complex multiplication and conjugation as follows: $\forall v \in V ,\ \alpha, \beta \in \mathbb{C}$, 
$$
\alpha(v \otimes \beta)=v \otimes(\alpha \beta)\quad\text{and}\quad  \overline{v \otimes \alpha}=v \otimes \overline{\alpha}. 
$$
Thus, the real vector space $V$ is naturally contained in the complex vector space $V_{\mathbb{C}}$ via the map $v \mapsto v \otimes 1$. 

Suppose that $V$ is endowed with an almost complex structure $I$, and we will also denote by $I$ its $\mathbb{C}$-linear extension to an endomorphism $V_{\mathbb{C}} \rightarrow V_{\mathbb{C}}$, 
given by $I\otimes \operatorname{id}_\mathbb{C}$. Clearly, the only eigenvalues of $I$ on $V_{\mathbb{C}}$ are $\pm i$. Then the $\pm i$ eigenspaces are denoted $V^{1,0}$ and $V^{0,1}$, respectively, i.e.
$$
V^{1,0}=\left\{v \in V_{\mathbb{C}} \mid I(v)=i \cdot v\right\} \text { and } V^{0,1}=\left\{v \in V_{\mathbb{C}} \mid I(v)=-i \cdot v\right\} .
$$
\begin{lemma}
  Let $V$ be a real vector space endowed with an almost complex structure $I$. Then
  $$
  V_{\mathbb{C}}=V^{1,0} \oplus V^{0,1}
  $$
  Complex conjugation on $V_{\mathbb{C}}$ induces an $\mathbb{R}$-linear isomorphism $V^{1,0} \cong V^{0,1}$.
\end{lemma}
\begin{proof}[Proof]
  Since $V^{1,0} \cap V^{0,1}=0$, the canonical map $V^{1,0} \oplus V^{0,1} \longrightarrow V_{\mathbb{C}}$ is injective. 
  The first assertion follows from the existence of the inverse map
  $$
  v \longmapsto \frac{1}{2}(v-i I(v)) \oplus \frac{1}{2}(v+i I(v)) .
  $$

  For the second assertion we write $v \in V_{\mathbb{C}}$ as $v=x+i y$ with $x, y \in V$.
  Then 
  $$
  \begin{aligned}
    \overline{(v-i I(v))}&=\overline{I(x\otimes 1 + y\otimes i)-i I(x\otimes 1 + y\otimes i)} \\
    &=\overline{I(x)\otimes 1+I(y)\otimes i-I(x)\otimes i + I(y)\otimes 1} \\
    &=(x-i y+i I(x)+I(y))=(\bar{v}+i I(\bar{v})).
  \end{aligned}
  $$ Hence, complex conjugation interchanges the two factors.
\end{proof}
From the above proof, one easily knows that
\begin{equation}
  V^{1,0}=\left\{v-iI(v) \mid v\in V\right\} \text { and } V^{0,1}=\left\{v+iI(v) \mid v\in V\right\}.
\end{equation}
Also, $V^{1,0}$ and $V^{0,1}$ are both $n$-dimensional complex vector spaces. Therefore the matrix of $I$ in an appropriate basis here is diagonal, with $n$ entries equal to $i$, 
and $n$ entries equal to $-i$. 

One should be aware of the existence of two almost complex structures on $V_{\mathbb{C}}$. 
One is given by $I$ and the other one by $i$, denoted by $(V_\mathbb{C},I)$, respectively, $(V_\mathbb{C},i)$. 
Together with the complex structure given in Lemma 1, where a real vector space is made into a complex vector space $(V,I)$, we now have four complex vector spaces in total.

$(V,I)$, discussed above, is equivalent to $(V,i)$, and in any basis, the matrix of $I$ is an $n\times n$ diagonal matrix, with every diagonal entry equal to $i$, i.e.
$I$ satisfies the linear polynomial $I-i=0$.

As for in the complexification $V_\mathbb{C}$, the operator $I$ is a particular $2n\times 2n$ diagonalisable operator with eigenvalues $\pm i$ since satisfying the polynomial $I^2+1=0$.
And $i$ simply acts as the scalar multiplication operator on $V_\mathbb{C}$, written by $i\cdot \operatorname{id}_{2n\times 2n}$. 
The two almost complex structures on $V_{\mathbb{C}}$ coincide on the subspace $V^{1,0}$ but differ by a sign on $V^{0,1}$. In the sequel, we will always 
regard $V_{\mathbb{C}}$ as the complex vector space with respect to $i$. The $\mathbb{C}$-linear extension of $I$ is the additional structure that gives rise to the above decomposition. 

\subsection*{The fundamental form on vector spaces}

Let $(V,\langle\ ,\ \rangle)$, be a Euclidian vector space, i.e. $V$ is a real vector space and $\langle\ ,\ \rangle$ is a positive definite symmetric bilinear form.

\begin{definition}[Compatible]
  An almost complex structure $I$ on $V$ is compatible with the scalar product $\langle\ ,\ \rangle$ if $\langle I(v), I(w)\rangle=\langle v, w\rangle$ for all $v, w \in V$, 
  i.e. $I \in \mathrm{O}(V,\langle\ ,\ \rangle)$. 
\end{definition}

Denote $(V,\langle\ ,\ \rangle, I)$ the Euclidian vector space $(V,\langle\ ,\ \rangle)$ endowed by a compatible almost complex structure $I$. One can define the fundamental form.
\begin{definition}[Fundamental form]
  The fundamental form associated to $(V,\langle\ ,\ \rangle, I)$, is the form
  $$
  \omega:=-\langle(\ ), I(\ )\rangle=\langle I(\ ),(\ )\rangle .
  $$
\end{definition}
One can know that $\omega$ is skew-symmetric since, for all $v, w \in V$, 
$$
\omega(v,w)=-\langle v, I(w)\rangle=-\langle I(w), v\rangle=\langle I(w), I(I(v))\rangle=\langle w, I(v)\rangle=-\omega(w,v).
$$

Note that two of the three structures $\{\langle\ ,\ \rangle, I, \omega\}$ determine the remaining one (by $I \in \operatorname{GL}(V)$).

By lemma 1, we know that $(V,I)$ is a complex vector space, and a natural Hermitian form on $(V,I)$ can be given by the scalar product and the fundamental form. 
\begin{lemma}
  $(V,\langle\ ,\ \rangle, I)$ and $\omega$ are given as above. The form $(\ ,\ ):=\langle\ ,\ \rangle-i \cdot \omega$, 
  is a positive Hermitian form on $(V, I)$.
\end{lemma}
\begin{proof}[Proof]
  The form $(\ ,\ )$ is clearly $\mathbb{R}$-linear and $(v, v)=\langle v, v\rangle>0$ for $0 \neq v \in V$.
  Moreover, $(v, w)=\overline{(w, v)}$ and
  $$
  \begin{aligned}
  (I(v), w) &=\langle I(v), w\rangle-i \cdot \omega(I(v), w) \\
  &=\langle I(I(v)), I(w)\rangle+i \cdot\langle v, w\rangle \\
  &=i \cdot(i \cdot\langle v, I(w)\rangle+\langle v, w\rangle)=i \cdot(v, w) .
  \end{aligned}
  $$
\end{proof}

We will show that the fundamental form $\omega$ is of type $(1,0)$. To do so, we need to consider the extension of the scalar product, 
the dual space of $V_\mathbb{C}$ and its exterior algebra. 

One considers the extension of the scalar product $\langle\ ,\ \rangle$ to a positive definite Hermitian form $\langle\ ,\ \rangle_{\mathbb{C}}$ on $V_{\mathbb{C}}$. 
This is defined by
$$
\langle v \otimes \lambda, w \otimes \mu\rangle_{\mathbb{C}}:=(\lambda \bar{\mu}) \cdot\langle v, w\rangle
$$
for $v, w \in V$ and $\lambda, \mu \in \mathbb{C}$.

An easy calculation shows $\langle v-i I(v), w+i I(w)\rangle_{\mathbb{C}}=0,\ \forall v,w\in V$. Then we have
\begin{lemma}
  $V_{\mathbb{C}}=V^{1,0} \oplus V^{0,1}$ is an orthogonal decomposition with respect to $\langle\ ,\ \rangle_{\mathbb{C}}$. 
\end{lemma}

And by the way, there is a relation between $(\ ,\ )$ and $\langle\ ,\ \rangle_{\mathbb{C}}$. Under the canonical isomorphism $(V, I) \cong(V^{1,0}, i)$, given by $v \mapsto \frac{1}{2}(v-i I(v))$, 
one has 
$$
\frac{1}{2}(\ ,\ )=\langle\ ,\ \rangle_{\mathbb{C}}\big|_{V^{1,0}},
$$ 
for an easy calculation shows $\langle\big(v-i I(v)\big),\big(w-i I(w)\big)\rangle_{\mathbb{C}}=2\left(v, w\right)$. 

An almost complex structure on $V$, $I$, also induces an almost complex structure on the dual space of $V^*=\text{Hom}_\mathbb{R}(V,\mathbb{R})$ which is given by $I(f)(v)=f(I(v))$.

As the same as $V_\mathbb{C}$, we can consider the complexification of the dual space. The complexification of $V^{*}$ can naturally be thought of as the space of all real linear maps 
from $V$ to $\mathbb{C}$ (denoted $\operatorname{Hom}_{\mathbb{R}}(V, \mathbb{C})$ ). That is,
$$
\left(V^{*}\right)_{\mathbb{C}}=V^{*} \otimes \mathbb{C} \cong \operatorname{Hom}_{\mathbb{R}}(V, \mathbb{C})
$$
The isomorphism is given by
$$
\left(\varphi_{1} \otimes 1+\varphi_{2} \otimes i\right) \leftrightarrow \varphi_{1}+i \varphi_{2}
$$
where $\varphi_{1}$ and $\varphi_{2}$ are elements of $V^{*}$. A real linear map $\varphi: V \rightarrow \mathbb{C}$ can be uniquely extended by linearity to obtain a complex linear map 
$\varphi: V_{\mathbb{C}} \rightarrow \mathbb{C}$. That is, 
$$
\varphi(v \otimes z)=z \varphi(v).
$$
This extension gives an isomorphism from $\operatorname{Hom}_{\mathbb{R}}(V, \mathbb{C})$ 
to $\operatorname{Hom}_{\mathbb{C}}(V_{\mathbb{C}}, \mathbb{C})$ (surjective by letting $\varphi(v)=\varphi(v\otimes 1)$). The latter is just the complex dual space 
to $V_{\mathbb{C}}$, so we have a natural isomorphism: $\left(V^{*}\right)_{\mathbb{C}} \cong\left(V_{\mathbb{C}}\right)^{*}$ (denoted by $V_\mathbb{C}^*$). 

Then the induced decomposition on $V_\mathbb{C}^*$ is given by
\begin{align*}
  (V^*)^{1,0}=\{f\in V_\mathbb{C}^* \mid I(f)=i\cdot f\}=(V^{1,0})^*, \\
  (V^*)^{0,1}=\{f\in V_\mathbb{C}^* \mid I(f)=-i\cdot f\}=(V^{0,1})^*. 
\end{align*}
And we have (consider $I\big(f(v)\big)$) 
\begin{equation}
  \begin{aligned}
    (V^*)^{1,0}=\{f\in V_\mathbb{C}^* \mid f(v)=0,\ \forall v\in V^{0,1}\}, \\
    (V^*)^{0,1}=\{f\in V_\mathbb{C}^* \mid f(v)=0,\ \forall v\in V^{1,0}\}. 
  \end{aligned}
\end{equation}

We are about to discuss the exterior algebra of $V_\mathbb{C}^*$, but for convenience, here we are actually discussing about $V_\mathbb{C}$, and there is no essential difference. 
The exterior algebra of $V$ has a natural decomposition, denoted by
$$
\sideset{}{^*}\bigwedge V=\bigoplus_{k=0}^{2n}\sideset{}{^k}\bigwedge V.
$$
Analogously, $\bigwedge^* V_\mathbb{C}$ denotes the exterior algebra of the complex vector space $V_\mathbb{C}$, and the complex conjugation in $V_{\mathbb{C}}$, extended to $\bigwedge^* V_\mathbb{C}$ in a natural manner

One defines $\bigwedge^{p, q} V$ the subspace of $\bigwedge^* V_\mathbb{C}$ spanned by $\alpha\wedge\beta$,
where $\alpha\in\bigwedge^p V^{1,0}$ and $\beta\in\bigwedge^q V^{0,1}$. The following proposition is evident. 
\begin{proposition}
  For a real vector space $V$ endowed with an almost complex structure $I$ one has:
  
  i) $\bigwedge^{k} V_{\mathbb{C}}=\bigoplus_{p+q=k} \bigwedge^{p, q} V$.

  ii) Complex conjugation on $\bigwedge^{*} V_{\mathbb{C}}$ defines a ($\mathbb{C}$-antilinear) isomorphism $\bigwedge^{p, q} V \cong \bigwedge^{q, p} V$, 
  i.e. $\overline{\bigwedge^{p, q} V}=\bigwedge^{q, p} V$.
\end{proposition}
\begin{proof}[Proof]
  Since $V$ has real dimension $2n$, $V_{\mathbb{C}}$ decomposes as $V_{\mathbb{C}}=V^{1,0} \oplus V^{0,1}$ with $V^{1,0}$ and $V^{0,1}$ complex vector spaces of dimension $n$. 

  Let $v_{1}, \cdots, v_{n} \in \bigwedge^{1,0} V=V^{1,0}$ and $w_{1}, \cdots, w_{n} \in \bigwedge^{0,1} V=V^{0,1}$ be $\mathbb{C}$-basis. Then $v_{J_{1}} \otimes w_{J_{2}} \in \bigwedge^{p, q} V$ with 
  $J_{1}=\left\{i_{1}<\cdots<i_{p}\right\}$ and $J_{2}=\left\{j_{1}<\right.$ $\left.\cdots<j_{q}\right\}$ form a basis of $\bigwedge^{p, q} V$.

  This shows i). 

  Since complex conjugation is multiplicative, i.e. $\overline{w_{1} \wedge w_{2}}=\overline{w_{1}} \wedge \overline{w_{2}}$, assertion ii) follows from $\overline{V^{1,0}}=V^{0,1}$. 
\end{proof}

In the same way, one can also define the exterior algebra $\bigwedge^{p, q} V^*$ and have the corresponding proposition as above. 

By the fundamental form $\omega$ is skew-symmetric, one has $\omega \in \bigwedge^{2} V^{*}$. Since $\bigwedge^{2} V^{*}$ can be considered as a subspace of $\bigwedge^{2} V^{*}_\mathbb{C}$, $\omega$ may be considered as 
an element of $\bigwedge^{2} V^{*}_\mathbb{C}$, i.e. $\omega$ can be uniquely extended to a bilinear form on $V_\mathbb{C}$, denoted also by $\omega$. 
And $\omega$ preserves the almost complex structure, i.e. $\omega(u,v)=\omega\bigl(I(u),I(v)\bigr)$, $\forall u,v\in V_\mathbb{C}$.
Then one knows that $\omega \in \bigwedge^{1,1} V^{*}$.  
\begin{proposition}
  Given $(V,\langle\ ,\ \rangle, I)$, then its fundamental form $\omega$ is a real $(1,1)$ form, 
  i.e. $\omega \in \bigwedge^{2} V^{*} \cap \bigwedge^{1,1} V^{*}$.
\end{proposition}
\begin{proof}[Proof]
  % Since  
  % $$
  % \begin{aligned}
  %   & \omega(v+iI(v),w+iI(w))=\langle I(v)-iv,\overline{w+iI(w)}\rangle_\mathbb{C}=-i\cdot\langle v+iI(v),\overline{w}-iI(\overline{w})\rangle_\mathbb{C}=0, \\
  %   & \omega(v-iI(v),w-iI(w))=\langle I(v)+iv,\overline{w-iI(w)}\rangle_\mathbb{C}= i\cdot\langle v-iI(v),\overline{w}+iI(\overline{w})\rangle_\mathbb{C}=0,
  % \end{aligned}
  % $$
  % for all $v,w\in V$. Then by (1) and (2) we have $\omega\in\bigwedge^{1,1} V^{*}$.
  By (1), one can calculate that $\omega$, and by (2) it is of $(1,1)$ type for it preserves the almost complex structure. 
  is real. 
\end{proof}

Since $\overline{I(v)}=I(\bar{v})$, $\forall v\in V_{\mathbb{C}}$ (similarly in Lemmma 2), we have: for all $v,w\in V_{\mathbb{C}}$,
$$
\omega(v,w)=\langle I(v),\overline{w}\rangle_\mathbb{C}=-\langle v,I(\overline{w})\rangle_\mathbb{C}.
$$
Here we calculate the extension of $\omega$ under given coordinate of $V_\mathbb{C}$. 

Let $z_{1}, \cdots, z_{n}$ be a $\mathbb{C}$-basis of $V^{1,0}$, and let $z^{1}, \cdots, z^{n}$ be a $\mathbb{C}$-basis of $(V^*)^{1,0}$. 
% Write $z_{i}=\frac{1}{2}\left(x_{i}-i I\left(x_{i}\right)\right)$ with $x_{i} \in V$. 
% Then $x_{1}, y_{1}:=I\left(x_{1}\right), \cdots, x_{n}, y_{n}:=I\left(x_{n}\right)$ is a $\mathbb{R}$-basis of $V$ and $x_{1}, \cdots, x_{n}$ is a $\mathbb{C}$-basis of $(V, I)$. 
The Hermitian form $\langle\ ,\ \rangle_{\mathbb{C}}$ on $V^{1,0}$ is given by an Hermitian matrix $\left(h_{j \bar{k}}\right)_{n\times n}$. Concretely, 
$$
\left\langle\sum_{j=1}^{n} a_{j} z_{j}, \sum_{k=1}^{n} b_{k} z_{k}\right\rangle_{\mathbb{C}}=\sum_{j, k=1}^{n} h_{j \bar{k}} a_{j} \bar{b}_{k} .
$$
% Using the lemma above, we obtain $\left(x_{i}, x_{j}\right)=h_{i j}$. Since $(\ ,\ )$ is Hermitian on $(V, I)$, this yields $\left(x_{i}, y_{j}\right)=-i h_{i j}$ and $\left(y_{i}, y_{j}\right)=h_{i j}$.

% By definition of $(\ ,\ )$, one has $\omega=-\operatorname{Im}(\ ,\ )$, and $\langle\ ,\ \rangle=\operatorname{Re}(\ ,\ )$. Hence, 
% $\omega\left(x_{i}, x_{j}\right)=\omega\left(y_{i}, y_{j}\right)=-\operatorname{Im}\left(h_{i j}\right), \omega\left(x_{i}, y_{j}\right)=\operatorname{Re}\left(h_{i j}\right),
% \left\langle x_{i}, x_{j}\right\rangle=\left\langle y_{i}, y_{j}\right\rangle=\operatorname{Re}\left(h_{i j}\right)$, 
% and $\left\langle x_{i}, y_{j}\right\rangle=\operatorname{Im}\left(h_{i j}\right)$. Thus,
% $$
% \omega=-\sum_{i<j} \operatorname{Im}\left(h_{i j}\right)\left(x^{i} \wedge x^{j}+y^{i} \wedge y^{j}\right)+\sum_{i, j=1}^{n} \operatorname{Re}\left(h_{i j}\right) x^{i} \wedge y^{j}
% $$
% Using $z^{i} \wedge \bar{z}^{j}=\left(x^{i}+i y^{i}\right) \wedge\left(x^{j}-i y^{j}\right)=x^{i} \wedge x^{j}-i\left(x^{i} \wedge y^{j}+x^{j} \wedge y^{i}\right)+y^{i} \wedge y^{j}$ this yields

For all $v,w\in V_\mathbb{C}$, suppose $v=z^j(v)z_j+\overline{z}^j(v)\overline{z}_j$, $w=z^j(w)z_j+\overline{z}^j(w)\overline{z}_j$. Then we have
$$
\begin{aligned}
  \omega(v,w)
    &=\langle I(v),\overline{w}\rangle_\mathbb{C} \\
    &=\sum_{j, k}\langle I\big(z^j(v)z_j+\overline{z}^j(v)\overline{z}_j\big),z^k(w)z_k+\overline{z}^k(w)\overline{z}_k\rangle_\mathbb{C} \\
    &=i \sum_{j, k}h_{j\bar{k}}\big(z^j(v)\overline{z}^k(w)-z^j(w)\overline{z}^k(v)\big)
\end{aligned}
$$
Then we have
$$
\omega=i \sum_{j, k}h_{j \bar{k}} z^{j} \wedge \bar{z}^{k}.
$$

% If $x_{1}, y_{1}, \cdots, x_{n}, y_{n}$ is an orthonormal basis of $V$ with respect to $\langle\ ,\ \rangle$ , i.e. $\langle\ ,\ \rangle=\sum_{i=1}^{n} x^{i} \otimes x^{i}+\sum_{i=1}^{n} y^{i} \otimes y^{i}$, then
% $$
% \omega=\frac{i}{2} \sum_{i=1}^{n} z^{i} \wedge \bar{z}^{i}=\sum_{i=1}^{n} x^{i} \wedge y^{i}
% $$

% Note that there always exists an orthonormal basis as above. Indeed, pick $x_{1} \neq 0$ arbitrary of norm one and define $y_{1}=I\left(x_{1}\right)$, which is automatically orthogonal to $x_{1}$. 
% Then continue with the orthogonal complement of $x_{1} \mathbb{R} \oplus y_{1} \mathbb{R}$.

\subsection*{Complex and almost complex manifold}

With enough preparation, we introduce complex and almost complex manifold manifolds, and then Hermitian manifolds which is exactly what we are trying to explain.
\begin{definition}[Holomorphically compatible]
    Let $X$ be a differentiable manifold. A complex chart on $X$ is a homeomorphism $\varphi: U \rightarrow V$ of an open subset $U \subset X$ onto an open subset $V \subset \mathbb{C}^n$. 
    Two complex charts $\varphi_{i}: U_{i} \rightarrow V_{i}, i=1,2$ are said to be holomorphically compatible if the maps
    $$
    \varphi_{2} \circ \varphi_{1}^{-1}: \varphi_{1}\left(U_{1} \cap U_{2}\right) \rightarrow \varphi_{2}\left(U_{1} \cap U_{2}\right)
    $$
    $$
    \varphi_{1} \circ \varphi_{2}^{-1}: \varphi_{2}\left(U_{1} \cap U_{2}\right) \rightarrow \varphi_{1}\left(U_{1} \cap U_{2}\right)
    $$
    are holomorphic.
\end{definition}
A complex atlas on $X$ is a system $\mathfrak{A}=\left\{\varphi_{i}: U_{i} \rightarrow V_{i}, i \in I\right\}$ of charts which are holomorphically compatible and which cover $X$, i.e., 
$\bigcup_{i \in I} U_{i}=X$.

Two complex atlases $\mathfrak{A}$ and $\mathfrak{A}^\prime$ on $X$ are called analytically equivalent if every chart of $\mathfrak{A}$ is holomorphically compatible with 
every chart of $\mathfrak{A}^{\prime}$.

\begin{definition}[Complex manifold]
    A complex manifold $X$ of dimension $n$ is a $2n$-dimensional differential manifold endowed with an equivalence class of analytically equivalent atlases. 
\end{definition}

We will see that a complex structure is determined by an almost complex structure on any tangent space, 
a purely linear algebra notion, satisfying a certain integrability condition.

\begin{definition}[Almost complex manifold]
  An almost complex manifold is a differentiable manifold $X$ together with a vector bundle endomorphism
  $$
  I: TX\rightarrow TX, \text{ with }I^2 = -\operatorname{id}. 
  $$
  Here, $T X$ is the real tangent bundle of the underlying real manifold.
\end{definition}

The endomorphism is also called the almost complex structure on the underlying differentiable manifold. If an almost complex structure exists, 
then the real dimension of $X$ is even. 

To show that every complex manifold carries a natural almost complex structure, we consider the space $\mathbb{C}^{n}$ of $n$-tuples of complex numbers $\left(z^{1}, \cdots, z^{n}\right)$ 
with $z^{j}=x^{j}+i y^{j}, j=1, \cdots, n$. With respect to the coordinate system $\left(x^{1}, \cdots, x^{n}, y^{1}, \cdots, y^{n}\right)$ we define an almost complex structure $I$ 
on each tangent space $\mathbb{C}^{n}$ by
$$
I\left(\frac{\partial}{\partial x^{j}} \right)=\frac{\partial}{\partial y^{j}} , \quad I\left(\frac{\partial}{\partial y^{j}} \right)=-\frac{\partial}{\partial x^{j}}, \quad j=1, \cdots, n.
$$

\begin{lemma}
  A mapping $f$ of an open subset of $\mathbb{C}^{n}$ into $\mathbb{C}^{m}$ preserves the almost complex structures of $\mathbb{C}^{n}$ and $\mathbb{C}^{m}$, 
  i.e., $f_{*} \circ I_{\mathbb{C}^{n}}=I_{\mathbb{C}^{m}} \circ f_{*}$, if and only if $f$ is holomorphic.
\end{lemma}
\begin{proof}[Proof]
  Let $\left(w^{1}, \cdots, w^{m}\right)$ with $w^{k}=u^{k}+i v^{k}, k=1, \cdots, m$, be the natural coordinate system in $\mathbb{C}^{m}$. 
  Express $f$ in terms of these coordinate systems in $\mathbb{C}^{n}$ and $\mathbb{C}^{m}$: for $k=1, \cdots, m,$
  $$
  \begin{aligned}
  &u^{k}=u^{k}\left(x^{1}, \cdots, x^{n}, y^{1}, \cdots, y^{n}\right), \\
  &v^{k}=v^{k}\left(x^{1}, \cdots, x^{n}, y^{1}, \cdots, y^{n}\right),
  \end{aligned}
  $$
  then $f$ is holomorphic when and only when the following Cauchy-Riemann equations hold:
  $$
  \frac{\partial u^{k}}{\partial x^{j}} - \frac{\partial v^{k}}{\partial y^{j}} =0\quad\text{and}\quad\frac{\partial u^{k}}{\partial y^{j}} + \frac{\partial v^{k}}{\partial x^{j}} =0,
  % \begin{aligned}
  % &, \\
  % &
  % \end{aligned}
  $$
  where $j=1, \cdots, n ; \ k=1, \cdots, m .$

  On the other hand, we have (whether $f$ is holomorphic or not), for $j=1, \cdots, n$, 
  $$
  \begin{gathered}
  f_{*}\left(\partial / \partial x^{j}\right)=\sum_{k=1}^{m}\left(\partial u^{k} / \partial x^{j}\right)\left(\partial / \partial u^{k}\right)+\sum_{k=1}^{m}\left(\partial v^{k} / \partial x^{j}\right)\left(\partial / \partial v^{k}\right), \\
  f_{*}\left(\partial / \partial y^{j}\right)=\sum_{k=1}^{m}\left(\partial u^{k} / \partial y^{j}\right)\left(\partial / \partial u^{k}\right)+\sum_{k=1}^{m}\left(\partial v^{k} / \partial y^{j}\right)\left(\partial / \partial v^{k}\right). 
  \end{gathered}
  $$

  From these formulas and the almost complex structure in $\mathbb{C}^{n}$ and $\mathbb{C}^{m}$ given above, we see that $f_{*} \circ I_{\mathbb{C}^{n}}=I_{\mathbb{C}^{m}} \circ f_{*}$ 
  if and only if $f$ satisfies the Cauchy-Riemann equations.
\end{proof}
To define an almost complex structure on a complex manifold $X$, we transfer the almost complex structure of $\mathbb{C}^n$ to $X$ by means of charts. Lemma 5 implies that an almost complex structure 
can be thus defined on $X$ independently of the choice of charts. 
An almost complex structure $I$ on a manifold $X$ is also called a complex structure 
if $X$ is an underlying differentiable manifold of a complex manifold which induces $I$ in the way just described.

\subsection*{Hermitian structure}

% The dual basis of $(T_xM)^*$ is denoted by $\mathrm{d}x^1, \cdots , \mathrm{d}x^n$, $\mathrm{d}y^1, \cdots, \mathrm{d}y^n.$ Recall that the induced almost complex structure on $T_xM$ in terms of this dual basis 
% is described by $I(\mathrm{d}x^i) = -\mathrm{d}y^i, I(\mathrm{d}y^i) = \mathrm{d}x^i$. 

To study the tangent vector bundle and its complexification, we first introduce the holomorphic vector bundle, a complex vector bundle with a holomorphic structure.
\begin{definition}
  Let $X$ be a complex manifold. A holomorphic vector bundle of rank $r$ on $X$ consists of a complex manifold $E$ together with a holomorphic map $\pi: E \longrightarrow X$ and 
  the structure of an $r$-dimensional complex vector space on any fibre $\pi^{-1}(x)$, satisfying the following condition: there exists an open covering $X=\bigcup U_i$ and biholomorphic maps 
  $$
  \psi_i: \pi^{-1}\left(U_i\right) \cong U_i \times \mathbb{C}^r
  $$ 
  commuting with the projections to $U_i$ such that the induced map $\pi^{-1}(x) \cong \mathbb{C}^r$ is $\mathbb{C}$-linear.  
\end{definition}

Let $X$ be an almost complex manifold. Then $T_{\mathbb{C}} X$ denotes the complexification of $T X$, i.e. $T_{\mathbb{C}} X=T X \otimes \mathbb{C}$. 
We emphasize that even for a complex manifold $X$ the bundle $T_{\mathbb{C}} X$ is a priori just a complex vector bundle without a holomorphic structure.
\begin{proposition}
  i) Let $X$ be an almost complex manifold. Then there exists a direct sum decomposition
  $$
  T_{\mathbb{C}} X=T^{1,0} X \oplus T^{0,1} X
  $$
  of complex vector bundles on $X$, such that the $\mathbb{C}$-linear extension of $I$ acts as multiplication by $i$ on $T^{1,0} X$ respectively by $-i$ on $T^{0,1} X$.

  ii) If $X$ is a complex manifold, then $T^{1,0} X$ is the holomorphic vector bundle on $X$ of rank $n$ which is 
  given by the transition functions $J\left(\varphi_i \circ \varphi_j^{-1}\right) \circ \varphi_j$.
\end{proposition}
\begin{proof}[Proof]
  i) One defines $T^{1,0} X$ and $T^{0,1} X$ as the kernel of $I-i \cdot$ id respectively $I+i$. id. That these maps are vector bundle homomorphisms and that 
  the direct sum decomposition holds, follows from the direct sum decomposition on all the fibres (cf. Lemma 2)
  
  ii) Giving a holomorphic map $f: U \rightarrow V$ between open subsets $U \subset \mathbb{C}^m$ and $V \subset \mathbb{C}^n$, one knows the $\mathbb{C}$-linear extension 
  of the differential $f_*: T_x U \rightarrow T_{f(x)} V$ respects the decomposition, 
  i.e. $f_*\left(T_x^{1,0} U\right) \subset T_{f(x)}^{1,0} V$ and $f_*\left(T_x^{0,1} U\right) \subset T_{f(x)}^{0,1} V$. 

  Moreover, the Jacobian
  $$
  f_*: T_x^{1,0} U \oplus T_x^{0,1} U \cong T_{f(x)}^{1,0} V \oplus T_{f(x)}^{0,1} V
  $$
  of a biholomorphic map $f: U \cong V$, where $U, V \subset \mathbb{C}^n$ are open subsets, has the form 
  $$
  \left(\begin{array}{cc}J(f) & 0 \\ 0 & \overline{J(f)}\end{array}\right).
  $$

  Let $X=\bigcup U_i$ be a covering by holomorphic charts $\varphi_i: U_i \cong \varphi\left(U_i\right)=V_i\subset\mathbb{C}^n$. 
  Then $\left(\varphi_i^{-1}\right)^*\left(\left.T^{1,0} X\right|_{U_i}\right) \cong T^{1,0} V_i$ and the latter is canonically trivialized. 
  With respect to these canonical trivializations the induced isomorphisms $T_{\varphi_j(x)}^{1,0} V_j \cong T_{\varphi_i(x)}^{1,0} V_i$ are given by 
  $J\left(\varphi_i \circ \varphi_j^{-1}\right) \circ \varphi_j(x)$. 
\end{proof}

\begin{definition}[Holomorphic and antiholomorphic tangent bundle]
  The bundles $T^{1,0} X$ and $T^{0,1} X$ are called the holomorphic respectively the antiholomorphic tangent bundle of the (almost) complex manifold $X$.
\end{definition}


As in the vector space situation, we are more interested in the dual bundles. We can use the results of the previous section in order to decompose the bundles of $k$-forms.

\begin{definition}
  For an almost complex manifold $X$ one defines the complex vector bundles $\bigwedge^k X:=\bigwedge^k\left(T X\right)^*$ and 
  $\bigwedge_\mathbb{C}^k X:=\bigwedge^k\left(T_{\mathbb{C}} X\right)^*$
  $$
  \sideset{}{^{p, q}}\bigwedge X:=\{\alpha\wedge\beta\mid\alpha\in\sideset{}{^{p}}\bigwedge\left(T^{1,0} X\right)^*\text{ and }\beta\in\sideset{}{^{q}}\bigwedge\left(T^{0,1} X\right)^*\}. 
  $$
  By $\mathcal{A}^k(X)$, $\mathcal{A}_{\mathbb{C}}^k(X)$ and $\mathcal{A}^{p, q}(X)$ we denote the spaces of sections of $\bigwedge^k X$, $\bigwedge_{\mathbb{C}}^k X$ and 
  $\bigwedge^{p, q} X$, respectively. Elements in $\mathcal{A}^{p, q}(X)$ are called forms of type $(p, q)$.
\end{definition}

As immediate consequences of Proposition 1, there exists a natural direct sum decomposition
$$
\sideset{}{_\mathbb{C}^k}\bigwedge X=\bigoplus_{p+q=k} \sideset{}{^{p, q}}\bigwedge X. 
$$
Moreover, $\overline{\bigwedge^{p, q} X}=\bigwedge^{q, p} X$.

% \begin{definition}
%   Let $X$ be an almost complex manifold. If $d: \mathcal{A}_{X, \mathbb{C}}^k \rightarrow \mathcal{A}_{X, \mathbb{C}}^{k+1}$ is the $\mathbb{C}$-linear extension of the exterior differential, 
%   then one defines
%   $$
%   \partial:=\Pi^{p+1, q} \circ d: \mathcal{A}_X^{p, q} \longrightarrow \mathcal{A}_X^{p+1, q}, \bar{\partial}:=\Pi^{p, q+1} \circ d: \mathcal{A}_X^{p, q} \longrightarrow \mathcal{A}_X^{p, q+1}.
%   $$
% \end{definition}
% As in the proof of Lemma 1.3.6 the Leibniz rule for the exterior differential $d$ implies the Leibniz rule for $\partial$ and $\bar{\partial}$, 
% e.g. $\partial(\alpha \wedge \beta)=\partial(\alpha) \wedge \beta+(-1)^{p+q} \alpha \wedge \partial(\beta)$ for $\alpha \in \mathcal{A}^{p, q}(X)$.

We now go back to our definition 1 (definition of Hermitian structure). The metric $g$ is compatible with the natural (almost) complex structure on $X$ if for any $x \in X$ the scalar product $g_{x}$ on $T_{x} X$ 
is compatible with the induced almost complex structure $I$, i.e. $g_{x}(v, w)=g_{x}(I(v), I(w))$ for all $v, w \in T_{x} X$. By definition 4, one has in this situation a $(1,1)$-form 
$\omega \in \mathcal{A}^{1,1}(X) \cap \mathcal{A}^{2}(X)$ defined by $\omega:=g(I(\ ),(\ )),$ which is called the fundamental form of $g$. 

The complex manifold $X$ endowed with an Hermitian structure $g$ is called an Hermitian manifold. Note that the Hermitian structure $g$ is uniquely determined by 
the almost complex structure $I$ and the fundamental form $\omega$. Indeed, $g(\ ,\ )=\omega(\ , I(\ ))$.

One could as well define Hermitian structures on almost complex manifolds. All assertions concerning the linear structure would still be valid 
in this more general context. But, as soon as we use the splitting of the exterior differential $d=\partial+\bar{\partial}$, 
we need an integrable almost complex structure (for explanation of the splitting and an integrable almost complex structure, see \cite{Huybrechts2005}, section 1.3 and 2.6).

In local coordinates $z^1, \ldots, z^n$, the Hermitian metric $g$ can be extended uniquely 
to a complex symmetric bilinear form, determined by the components $g_{j \bar{k}}$ where
$$
g_{j \bar{k}}=g\left(\frac{\partial}{\partial z^j}, \frac{\partial}{\partial \bar{z}^k}\right).
$$
The Hermitian condition implies that for any $j, k$ we have
$$
g\left(\frac{\partial}{\partial z^j}, \frac{\partial}{\partial z^k}\right)=g\left(\frac{\partial}{\partial \bar{z}^j}, \frac{\partial}{\partial \bar{z}^k}\right)=0
$$
In terms of the components $g_{j \bar{k}}$ we can therefore write
$$
g=\sum_{j, k} g_{j \bar{k}}\left(d z^j \otimes d \bar{z}^k+d \bar{z}^k \otimes d z^j\right)
$$
Note that the bar on $\bar{k}$ in the components $g_{j \bar{k}}$ is used to remember the distinction between holomorphic and antiholomorphic components.

The symmetry of $g$ implies that $\overline{g_{j \bar{k}}}=g_{k \bar{j}}$, and the positivity of $g$ means that $g_{j \bar{k}}$ is a positive definite Hermitian matrix at each point. The associated 2-form $\omega$ can be written as
$$
\omega=i \sum_{j, k} g_{j \bar{k}} d z^j \wedge d \bar{z}^k
$$

% We can also consider the decomposition of the exterior algebra on tangent spaces. $T_{\mathbb{C}}X$ and $T^*_{\mathbb{C}}X$ stand for the complexification of tangent bundle and cotangent bundle. Then one defines the 
% complex vector bundles
% $$
% \sideset{}{^{p, q}}\bigwedge X:=\{\alpha\wedge\beta\mid\alpha\in\sideset{}{^{p}}\bigwedge\left(T^{*} X\right)^{1,0}\text{ and }\beta\in\sideset{}{^{q}}\bigwedge\left(T^{*} X\right)^{0,1}\}. 
% $$
% And by $\mathcal{A}_{\mathbb{C}}^{k}(X)$ and $\mathcal{A}^{p, q}(X)$ we denote the spaces of sections of $\bigwedge_{\mathbb{C}}^{k} X:=\bigwedge^{k} T_{\mathbb{C}}^{*} X$ and $\bigwedge^{p, q} X$, respectively.

% So far, we have thoroughly explained the definition of the Hermitian Structure. The complex manifold $X$ endowed with an Hermitian structure $g$ is called an Hermitian manifold. On a Hermitian manifold the metric can be written 
% in local holomorphic coordinates $(z^\alpha)$ as 
% $$ 
% h=h_{\alpha {\bar {\beta }}}\,dz^{\alpha }\otimes d{\bar {z}}^{\beta },
% $$
% where $h_{\alpha\bar\beta}$ are the components of a positive-definite Hermitian matrix. The metric $g$, as the real part, can be written as
% $$
% g={1 \over 2}h_{\alpha {\bar {\beta }}}\,\left(dz^{\alpha }\otimes d{\bar {z}}^{\beta }+d{\bar {z}}^{\beta }\otimes dz^{\alpha }\right).
% $$
% And the fundamental form, as the imaginary part, can be written as
% $$
% \omega ={i \over 2}h_{\alpha {\bar {\beta }}}\,dz^{\alpha }\wedge d{\bar {z}}^{\beta }.
% $$
% It is clear from the coordinate representations that any one of the three forms $h$, $g$, and $\omega$ uniquely determine the other two.

% All three forms $h$, $g$, and $\omega$ preserve the almost complex structure $I$. That is,
% $$
% \begin{aligned}
%   h(I(u),I(v))&=h(u,v)\\g(I(u),I(v))&=g(u,v)\\\omega (I(u),I(v))&=\omega (u,v)
% \end{aligned}
% $$
% for all $u,v\in TX$.
% An Hermitian structure on an (almost) complex manifold $X$ can therefore be specified by one of which: 
% \begin{itemize}
%   \item a Hermitian metric $h$ as above,
%   \item a Riemannian metric $g$ that preserves the almost complex structure $I$, 
%   \item a nondegenerate $2$-form $\omega$ which preserves $I$ and is positive-definite in the sense that $\omega\big(u, I(u)\big) > 0$ for all nonzero real tangent vectors $u$.
% \end{itemize}
% Note that many authors call $g$ itself the Hermitian metric.

At last, every paracompact (almost) complex manifold admits a Hermitian metric. 
\begin{proposition}
  Every almost complex manifold admits a Hermitian metric provided it is paracompact.
\end{proposition}
\begin{proof}[Proof]
  Given an arbitrary Riemannian metric $g$ (which exists provided $X$ is paracompact) on an almost complex manifold $X$ one can construct a new metric $g^\prime$ 
  compatible with the almost complex structure $I$ in an obvious manner:
  $$
  g^\prime(u,v)={1 \over 2}\big(g(u,v)+g(I(u),I(v))\big).
  $$
\end{proof}

% \paragraph*{Another definition of $\bigwedge^{p, q} V$}
\subsection*{* Another definition of $\bigwedge^{p, q} V$}

$\bigwedge^{p, q} V:=\bigwedge^p V^{1,0} \otimes_\mathbb{C} \bigwedge^q V^{0,1},$ where the exterior products of 
$V^{1,0}$ and $V^{0,1}$ are taken as exterior products of complex vector spaces. An element $\alpha \in \bigwedge^{p, q} V$ is of bidegree $(p, q)$. And one has the following propositions.

\begin{proposition}
  For a real vector space $V$ endowed with an almost complex structure $I$ one has:

  i) $\bigwedge^{p, q} V$ is in a canonical way a subspace of $\bigwedge^{p+q} V_{\mathbb{C}}$.

  ii) $\bigwedge^{k} V_{\mathbb{C}}=\bigoplus_{p+q=k} \bigwedge^{p, q} V$.

  iii) Complex conjugation on $\bigwedge^{*} V_{\mathbb{C}}$ defines a ($\mathbb{C}$-antilinear) isomorphism $\bigwedge^{p, q} V \cong \bigwedge^{q, p} V$, 
  i.e. $\overline{\bigwedge^{p, q} V}=\bigwedge^{q, p} V$.

  iv) The exterior product is, i.e. $(\alpha, \beta) \mapsto \alpha \wedge \beta$, maps $\bigwedge^{p, q} V \times \bigwedge^{r, s} V$ to the subspace $\bigwedge^{p+r, q+s} V$.
\end{proposition}
\begin{proof}[Proof]
  Here to show i) and ii), one could as well use the general fact that any direct sum decomposition $V_{\mathbb{C}}=W_{1} \oplus W_{2}$ induces 
  a direct sum decomposition $\bigwedge^{k} V_{\mathbb{C}}=\bigoplus_{p+q=k} \bigwedge^{p} W_{1} \otimes \bigwedge^{q} W_{2}$, given by the following.

  For vector spaces $V, W$ over a field $K$, we have
  $$
  \bigoplus_{k=0}^{n}\left(\sideset{}{^{k}}\bigwedge V \otimes_{K} \sideset{}{^{n-k}}\bigwedge W \right) \cong \sideset{}{^{n}}\bigwedge(V \oplus W), 
  $$
  which holds in a canonical way : for $0 \leq k \leq n$, the maps $\bigwedge^{k}(V) \times \bigwedge^{n-k}(W) \rightarrow \bigwedge^{n}(V \oplus W)$, given by 
  $$
  \left(v_{1} \wedge \cdots \wedge v_{k}, w_{k+1} \wedge \cdots \wedge w_{n}\right) \mapsto\left(v_{1} \wedge \cdots \wedge v_{k}\right) \wedge\left(w_{k+1} \wedge \cdots \wedge w_{n}\right).
  $$
  By universal properties of the tensor product, we obtain a map $\varphi_{k}: \bigwedge^{k}(V) \otimes_{K} \bigwedge^{n-k}(W) \rightarrow \bigwedge^{n}(V \oplus W)$, 
  and then get a canonical map (which is actually a bijective map)
  $$
  \varphi=\bigoplus_{k=0}^{n} \varphi_{k}: \bigoplus_{k=0}^{n}\left(\bigwedge^{k}(V) \otimes_{K} \bigwedge^{n-k}(W)\right) \longrightarrow \bigwedge^{n}(V \oplus W). 
  $$
\end{proof}

% With respect to the direct sum decompositions $\bigwedge^* V_\mathbb{C}=\bigoplus\limits_{k=0}^d\bigwedge^k V_\mathbb{C}$ and i) of Proposition 1 one defines the natural projections
% $$
% \Pi^{k}: \sideset{}{^*}\bigwedge V_{\mathbb{C}} \longrightarrow \sideset{}{^k}\bigwedge V_{\mathbb{C}} \text { and } \Pi^{p, q}: \sideset{}{^{*}}\bigwedge V_{\mathbb{C}} \longrightarrow \sideset{}{^{p, q}}\bigwedge V.
% $$
% Furthermore, $\mathbf{I}: \sideset{}{^*}\bigwedge V_{\mathbb{C}} \rightarrow \sideset{}{^*}\bigwedge V_{\mathbb{C}}$, a linear operator, is the multiplicative extension of 
% the almost complex structure $I$ on $V_{\mathbb{C}}$, defined as 
% $$
% \mathbf{I}=\sum_{p, q} i^{p-q} \cdot \Pi^{p, q} .
% $$
% The operator $\Pi^{k}$ does not depend on the almost complex structure $I$, but the operators $\mathbf{I}$ and $\Pi^{p, q}$ certainly do. Since $I$ is defined on the real vector space $V$, 
% also $\mathbf{I}$ is an endomorphism of the real exterior algebra $\bigwedge^{*} V$.

% We denote the corresponding operators on the dual space $\bigwedge^{*} V_{\mathbb{C}}^{*}$ also by $\Pi^{k}, \Pi^{p, q}$, respectively $\mathbf{I}$. Note that 
% $\mathbf{I}(\alpha)\left(v_{1}, \cdots, v_{k}\right)=\alpha\left(\mathbf{I}\left(v_{1}\right), \cdots, \mathbf{I}\left(v_{k}\right)\right)$ 
% for $\alpha \in \bigwedge^{k} V_{\mathbb{C}}^{*},\ v_{i} \in V_{\mathbb{C}}$.


\bibliographystyle{unsrt}
\bibliography{NoteRef}

\end{document}