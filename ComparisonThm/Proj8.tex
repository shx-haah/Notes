%!TEX program = Xelatex
\documentclass{ctexart}

\usepackage{amsmath, amssymb, amsthm, titlesec}
\usepackage[skins, breakable, theorems]{tcolorbox}
\usepackage{xparse, ocgx2, hyperref, endnotes}
\usepackage[top=2.54cm, bottom=2.54cm, left=3.18cm, right=3.18cm]{geometry}

\def\Id{\,\mathrm{d}} % 积分中的正体d
\newcommand{\norm}[1]{\left|#1\right|} % 范数

% [number within=section/...]{}{<display name>}{<style>}{<label>(cite as "Thm:...")}
\newtcbtheorem[]{tcbdefinition}{Definition}{fonttitle = \bfseries}{Def}
\newtcbtheorem[]{tcbtheorem}{Theorem}{fonttitle = \bfseries}{Thm}
\newtcbtheorem[]{tcbproposition}{Proposition}{fonttitle = \bfseries}{Pro}
\newtcbtheorem[]{tcblemma}{Lemma}{fonttitle = \bfseries}{Lem}
\newtcbtheorem[]{tcbcorollary}{Corollary}{fonttitle = \bfseries}{Cor}
\NewDocumentEnvironment{definition}{ O{} O{} } % two optional arguments
  {\tcbdefinition{#1}{#2}}
  {\endtcbdefinition}
\NewDocumentEnvironment{theorem}{ O{} O{} }
  {\tcbtheorem{#1}{#2}}
  {\endtcbtheorem}
\NewDocumentEnvironment{proposition}{ O{} O{} }
  {\tcbproposition{#1}{#2}}
  {\endtcbproposition}
\NewDocumentEnvironment{lemma}{ O{} O{} }
  {\tcblemma{#1}{#2}}
  {\endtcblemma}
\NewDocumentEnvironment{corollary}{ O{} O{} }
  {\tcbcorollary{#1}{#2}}
  {\endtcbcorollary}
\makeatletter
\newcommand\tcb@cnt@tcbdefinitionautorefname{Definition}
\newcommand\tcb@cnt@tcbtheoremautorefname{Theorem}
\newcommand\tcb@cnt@tcbpropositionautorefname{Proposition}
\newcommand\tcb@cnt@tcblemmaautorefname{Lemma}
\newcommand\tcb@cnt@tcbcorollaryautorefname{Corollary} 
\makeatother

% Sketch of proof & Remark
\newtcolorbox{sop}{
    blanker, breakable, left = 5mm,
    before skip = 10pt, after skip = 10pt,
    borderline west = {1mm}{0pt}{red},
    coltitle = black, title = \emph{Sketch of Proof}
}
\newtcolorbox{remark}{
    colback = white, fonttitle = \bfseries,
    colbacktitle = white, coltitle = black, enhanced,
    boxed title style = {sharp corners},
    attach boxed title to top left={yshift = -2mm, xshift = 5mm},
    title = Remark
}

\newcommand{\linktoprf}[1]{\hyperlink{#1}{To complete proof.}}
\newcommand{\targetprf}[1]{\hypertarget{#1}{Proof of \autoref{#1}}}

\title{Note 8: Comparison Theorems in Riemannian Geometry and Applications}
\date{} % 显示日期

\renewcommand\refname{Reference}

%%%%%%%%%%%%%%%%%%%%%%%%%%%%%%%%%%%%%%%%%%%%%%%%%%%%%
\begin{document}

\maketitle

In this Note, we first introduce some applications of Rauch Comparison Theorem. Then we will give another two Comparison Theorems, Hessian Comparison Theorem and Toponogov Comparison Theorem, 
and also some applications of these two theorems. 


\subsection*{Applications of Rauch Comparison Theorem}

Recall the Rauch Comparison Theorem in Note 7. By Hopf-Rinow Theorem, if $M$ is a complete Riemannian manifold with non-positive sectional curvature, $\exp_p$ is defined on the whole $T_pM$. 
Then for any $p \in M$, $\exp_{p}: T_pM \rightarrow M$ is a distance-expanding map. 
Furthermore, there is an interesting corollary, which can be found in \cite{WuHongXi2014}, Corollary 3, p.119.
\begin{corollary}
    Let $M$ be a simply connected complete Riemannian manifold with non-positive sectional curvature, and let $ABC$ be a geodesic triangle in $M$ 
    (i.e. all sides of the triangle are minimal geodesics), with corresponding angles $A, B, C$, and corresponding side $a, a, c$, then 

    (1) $a^2+b^2-2 a b \cos C \leq c^2$, 

    (2) $A+B+C \leq \pi$. 
\end{corollary}
\begin{proof}[Proof]
    Let $x$ be the vertex of angle $C$, and refer to Figure 8.2 in the tangent space $T_x M$. Draw radial line segments $\overline{O p}$ and $\overline{O q}$ with lengths $a$ and $b$, respectively, 
    such that $\exp_x$ maps $\overline{O p}$ and $\overline{O q}$ to the sides of the geodesic triangle with lengths $a$ and $b$. Let $\zeta$ be the $\exp_x^{-1}$ image of side $c$. 
    Since $\exp_x$ is a distance-expanding map, we have $L(\zeta) \leq c$, so the length of the line segment $\overline{p q}$ in $T_x M$ is $L(\overline{p q}) \leq L(\zeta) \leq c$.
    The law of cosines for the triangle $O p q$ in $T_x M$ is given by:
    $$
    a^2 + b^2 - 2ab\cos C = L(\overline{p q})^2,
    $$
    which proves (1).
    In $\mathbb{R}^2$, construct a triangle with boundary lengths $a$, $b$, and $c$. This is possible because the sides of a geodesic triangle are always the shortest geodesic paths, 
    and the sum of the lengths of any two sides must be greater than the length of the third side. Let $A', B', C'$ denote the angles of this triangle in $\mathbb{R}^2$.
    From (1), we know that $C \leq C'$. Similarly, we can deduce $A \leq A'$ and $B \leq B'$. Thus, (2) is proven.
\end{proof}
This corollary is a special case of Toponogov Comparison Theorem. The proof of the Toponogov theorem essentially relies on the Rauch Comparison theorem,
and the method is similar to the proof of this corollary. However, it typically involves conjugate points, making the situation more complex.

\begin{theorem}
  Let $M$ and $\tilde{M}$ be $m$-dimensional Riemannian manifolds, and let $K$ and $\tilde{K}$ be the sectional curvatures, respectively. Consider a point $p$ in $M$ and $\tilde{p}$ in $\tilde{M}$. 
  Let $\varphi: T_p M \rightarrow T_{\tilde{p}} \tilde{M}$ be an isometric linear isomorphism. Assume that $W$ and $\tilde{W}=\varphi(W)$ are neighborhoods of the origin in $T_p M$ and $T_{\tilde{p}} \tilde{M}$, 
  respectively, such that the exponential mappings
  $$
  \exp_p: W \rightarrow \exp_p(W) \subset M \quad \text{ and } \quad \exp_{\tilde{p}}: \tilde{W} \rightarrow \exp_{\tilde{p}}(\tilde{W}) \subset \tilde{M}
  $$
  are both diffeomorphisms. If for any corresponding points
  $$
  q \in \exp_p(W) \quad \text{ and } \quad \tilde{q}=\exp_{\tilde{p}} \circ \varphi \circ (\exp_q)^{-1}(q) \in \exp_{\tilde{p}}(\tilde{W}),
  $$
  and for any two-dimensional sections $\sigma \subset T_q M$ and $\tilde{\sigma} \subset T_{\tilde{q}} \tilde{M}$, the following inequality holds:
  $$
  K(\sigma) \geq \tilde{K}(\tilde{\sigma}),
  $$
  then, for any smooth curve $\beta(u)$ in $W$ with $0 \leq u \leq 1$, the following inequality holds:
  $$
  L(\exp_p \circ \beta) \leq L(\exp_{\tilde{p}} \circ (\varphi \circ \beta)).
  $$
\end{theorem}
\begin{proof}[Proof]
  Without loss of generality, we can assume that $\beta$ is nonzero everywhere (i.e., $\beta$ does not pass through the origin).
  Consider the geodesic variation:
  $$
  \alpha(t, u) = \exp_p(t \beta(u)), \quad 0 \leq t \leq 1, \quad 0 \leq u \leq 1.
  $$
  Let
  $$
  V(t, u) = \alpha_{*(t, u)}\left(\frac{\partial}{\partial u}\right) = \left(\exp_p\right)_{* t \beta(u)}\left(t \beta^{\prime}(u)\right).
  $$
  Then:
  $$
  \begin{aligned}
  V(1, u)  &= \left(\exp_p\right)_{* \beta(u)}\left(\beta^{\prime}(u)\right), \\
  L\left(\exp_p \circ \beta\right) &= \int_0^1|V(1, u)| \mathrm{d} u.
  \end{aligned}
  $$
  Similarly, we have:
  $$
  L\left(\exp_{\tilde{p}} \circ \varphi \circ \beta\right) = \int_0^1|\tilde{V}(1, u)| \mathrm{d} u,
  $$
  where:
  $$
  \tilde{V}(1, u) = \frac{\mathrm{d}}{\mathrm{d} u}\left(\exp_{\tilde{p}}(\varphi \circ \beta(u))\right) = \left(\exp_{\tilde{p}}\right)_{* \varphi \circ \beta(u)}\left(\varphi\left(\beta^{\prime}(u)\right)\right).
  $$

  For any fixed $u \in (0,1)$, consider the vector fields $V(t, u)$ and $\tilde{V}(t, u)$ along the geodesics $\gamma_u$ and $\tilde{\gamma}_u$, respectively, where
  $$
  \gamma_u(t) = \exp_p(t \beta(u)) \quad \text{ and } \quad \tilde{\gamma}_u(t) = \exp_{\tilde{p}}(t \varphi \circ \beta(u)). 
  $$
  According to the process of defining Jacobi fields, $V(t, u)$ and $\tilde{V}(t, u)$ are Jacobi fields along $\gamma_u$ and $\tilde{\gamma}_u$, respectively. By definition:
  $$
  \begin{aligned}
  & V(0, u) = \tilde{V}(0, u) = 0, \\
  & \frac{\nabla V}{\mathrm{d} t}(0, u) = \beta^{\prime}(u), \quad \frac{\nabla \tilde{V}}{\mathrm{d} t}(0, u) = \varphi\left(\beta^{\prime}(u)\right), \\
  & \gamma_u^{\prime}(0) = \beta(u), \quad \tilde{\gamma}_u^{\prime}(0) = \varphi(\beta(u)).
  \end{aligned}
  $$
  Based on the assumption that, for any $X \in T_{\gamma_u(t)} M$ and $\tilde{X} \in T_{\tilde{\gamma}_u(t)} \tilde{M}$:
  $$
  K\left(\gamma_u^{\prime}, X\right) \geq \tilde{K}\left(\tilde{\gamma}_u^{\prime}, \tilde{X}\right),
  $$
  we can apply Rauch Comparison Theorem, which implies:
  $$
  |V(t, u)| \leq |\tilde{V}(t, u)|.
  $$
  In particular:
  $$
  |V(1, u)| \leq |\tilde{V}(1, u)|,
  $$
  and thus:
  $$
  L\left(\exp_p \circ \beta\right) \leq L\left(\exp_{\tilde{p}} \circ \varphi \circ \beta\right).
  $$
\end{proof}
This corollary plays an essential role in the proof of the following Toponogov's Theorem and that is how the theorem relies on the Rauch Comparison theorem. 

\subsection*{Hessian Comparison Theorem and its application}
We will introduce the Hessian and Laplacian on manifolds as well as a naturally defined distance function.

Given $f\in C^2(M)$, the second-order covariant derivative of $f$, i.e. $\nabla^2 f = \nabla df$, is called the Hessian of $f$. 
By definition, for any smooth vector fields $X, Y$ on $M$, we have:
$$
\begin{aligned}
\nabla^2 f(X, Y) &= \nabla df(X, Y) \\
&= Y(Xf) - (\nabla_X Y)f.
\end{aligned}
$$
As $\nabla^2 f$ is a $(0,2)$-order symmetric tensor field, 
it induces a symmetric bilinear functional on $T_p M$ for each point $p$ in $M$.
The Laplacian of $f$, denoted as $\Delta f$, is equal to the trace of $\nabla^2(f)$, i.e.,
$$
\Delta f = \text{tr}(\nabla^2(f)) = \text{tr}(\nabla^2 f).
$$
If we choose a local coordinate system $(x^i)$ around a point $p \in M$, then by Proposition 3 in Note 6, we have:
$$
\begin{aligned}
\nabla^2 f\left(\frac{\partial}{\partial x^i}(p), \frac{\partial}{\partial x^j}(p)\right) &= \frac{\partial^2 f}{\partial x^i \partial x^j}(p), \\
\Delta f(p) &= \sum_i \frac{\partial^2 f}{\left(\partial x^i\right)^2}(p).
\end{aligned}
$$
From this, we can see that the value of $\nabla^2 f(X, Y)$ at point $p$ depends only on the values of the vectors $X$ and $Y$ at point $p$ and is independent of their behavior near $p$.

To avoid discussing the concepts of cut points and cut locus, we introduce another specialized definition: a geodesic is called "stably shortest" if all geodesics nearby it are also shortest. Specifically, let
$$
\begin{aligned}
  \widetilde{\gamma}:[0, a] &\rightarrow T_{\gamma(0)} M \\
  t &\mapsto t \dot{\gamma}(0)
\end{aligned}
$$
be a radial line segment with $\gamma(0)=x$. We call $\gamma=\exp _x \widetilde{\gamma}$ "stably shortest" if there exists a sector neighborhood $\mathcal{U}$ of $\widetilde{\gamma}$ in $T_x M$ such that:
$$
\mathcal{U}=\{t p\mid t \in[0, a],|p-\dot{\gamma}(0)|<\text { some constant }\}.
$$
All radial line segments in $\mathcal{U}$ are mapped to the shortest geodesics in $M$ starting from $x$ by $\exp _x$. 
From the continuity of the cut locus (see \cite{WuHongXi2014}, p.158, Lemma 5), one can show that 
$\gamma$ is stably shortest if and only if it does not contain any cut points along $\gamma$ away from $\gamma(0)$.

For any given point $x \in M$, we can define a distance function $\rho: M \rightarrow \mathbb{R}$ as follows: 
for any $y \in M$, $\rho(y)$ is defined as the distance between $x$ and $y$, denoted as $d(x, y)$. It is evident that the function $\rho$ is continuous, but it may not necessarily be differentiable.
If $\gamma:[0, a] \rightarrow M$ is a normal (with $|\gamma'|\equiv 1$) stably shortest geodesic, then it has a neighborhood $\exp _x \mathcal{U}$ (as described above) such that 
$\exp _x: \mathcal{U} \rightarrow \exp _x \mathcal{U}$ is a diffeomorphism, and
$$
\rho\left(\exp _x z\right)=|z|, \quad \forall z \in \mathcal{U} .
$$
Thus, $\rho$ is smooth on $\exp _x \mathcal{U}-\{x\}$. 
By Gauss's Lemma in Note 6, the gradient of the function $\rho$ is $\frac{\partial}{\partial \rho}$, i.e. for any tangent vector $X$ in $\exp _x \mathcal{U}-\{x\}$, 
$$
X (\rho)=\left\langle X, \frac{\partial}{\partial \rho}\right\rangle.
$$  

Before we state the Hessian Comparison Theorem, we give an important lemma on calculating the Hessian of distance functions. 
% Define a vector field $\frac{\partial}{\partial \rho}$ on $\exp _x \mathcal{U}-\{x\}$ as follows:
% For any $\exp _x z \in \exp _x \mathcal{U}-\{x\}$, we have
% $$
% \frac{z}{|z|} \in T_x M \approx  T_z\left(T_x M\right),
% $$
% so let
% $$
% \left.\frac{\partial}{\partial \rho}\right|_{\exp _x z}=\mathrm{d} \exp _x\left(\frac{z}{|z|}\right).
% $$
% It is clear from the definition of $\frac{\partial}{\partial \rho}$ that:
% \begin{equation}
%   \left\{
%   \begin{array}{l}
%     \frac{\partial}{\partial \rho} (\rho)=1, \\
%     \left\langle\frac{\partial}{\partial \rho}, \frac{\partial}{\partial \rho}\right\rangle=1 .
%   \end{array}
%   \right.  
% \end{equation}

% Using $\rho$ and $\frac{\partial}{\partial \rho}$, we can restate Gauss's Lemma as follows:
% \begin{lemma}
%   For any tangent vector $X$ in $\exp _x \mathcal{U}-\{x\}$, we have:
%   $$
%   X (\rho)=\left\langle X, \frac{\partial}{\partial \rho}\right\rangle.
%   $$  
% \end{lemma}
% \begin{proof}[Proof]
%   When $X=\frac{\partial}{\partial \rho}$, it is evident from equation (1) that the lemma holds. If
%   $$
%   \left\langle X, \frac{\partial}{\partial \rho}\right\rangle=0,
%   $$
%   then by Gauss's Lemma in Note 6, we know that $X$ is tangent to a geodesic sphere. Note that the distance function $\rho$ is constant on the geodesic sphere, so $X (\rho)=0$. This proves the lemma.  
% \end{proof}

\begin{lemma}
  Let $\gamma:[0,a]\rightarrow M$ be normal stably shortest geodesic and $J$ is a normal Jacobi field along $\gamma$ with $J(0)=0$. Then 
  $$
  \nabla^2\rho\left(J\left(a\right),J\left(a\right)\right) = \langle J'\left(a\right),J\left(a\right)\rangle. 
  $$
\end{lemma}
\begin{proof}[Proof]
  By Proposition 1 in Note 6, we know that Jacobi fields can be represented as transversal vector fields to a one-parameter family of geodesics. 
  Take $\varepsilon$ small enough so that for any $u\in [0,\varepsilon]$, $u J'(0)\in \mathcal{U}\subset T_{\gamma(0)} M$. 
  Define $\alpha : [0, a] \times [0, \varepsilon] \rightarrow M$, given by
  $$
  \alpha(t, u) =\exp_{\gamma(0)}\left(t\left(\gamma'(0)+uJ'(0)\right)\right).
  $$
  Let $T = \alpha_*\left(\frac{\partial}{\partial t}\right)$ and $V = \alpha_*\left(\frac{\partial}{\partial u}\right)$, and we have:
  $$
  V(\gamma(t)) = J(t) \quad \text{ and } \quad T(\gamma(t)) = \frac{\partial}{\partial \rho}\big|_{\gamma(t)}.
  $$
  Now, let's compute at $\gamma(a)$: 
  $$
  \begin{aligned}
  \nabla^2 \rho(J(a), J(a)) &= V V \rho - \left(\nabla_V V\right) \rho \\
  &= V\left\langle V, \frac{\partial}{\partial \rho}\right\rangle - \left\langle\nabla_V V, \frac{\partial}{\partial \rho}\right\rangle \\
  &= V\langle V, T\rangle - \left\langle\nabla_V V, T\right\rangle \\
  &= \left\langle\nabla_V V, T\right\rangle + \left\langle V, \nabla_V T\right\rangle - \left\langle\nabla_V V, T\right\rangle \\
  &= \left\langle\nabla_V T, V\right\rangle = \left\langle\nabla_T V, V\right\rangle  \\
  &= \langle\dot{J}(a), J(a)\rangle.
  \end{aligned}
  $$
  In the above calculation, we used Gauss's Lemma and $ [V, T] = 0$. Additionally, for simplicity, we did not explicitly indicate that the functions should be evaluated at $\gamma(a)$.
\end{proof}

Let $M, \widetilde{M}$ be $n$-dimensional , $\gamma:[0, a] \rightarrow M, \widetilde{\gamma}:[0, a] \rightarrow \widetilde{M}$ are normal geodesics and $x=\gamma(0), \widetilde{x}=\widetilde{\gamma}(0)$. 
Suppose $\rho, \widetilde{\rho}$ be distance functions on $M, \widetilde{M}$, associated with $x, \widetilde{x}$, respectively. 
\begin{theorem}[Hessian Comparison Theorem]
  Suppose 

  (1) $\gamma$ and $\widetilde{\gamma}$ are stably shortest; 

  (2) For all $t$ and all $x \in T_{\gamma(t)}(M), \tilde{x} \in T_{\tilde{\gamma}(t)}(\tilde{M})$, we have
  $$
  \tilde{K}\left(\tilde{x}, \tilde{\gamma}^{\prime}(t)\right) \geq K\left(x, \gamma^{\prime}(t)\right). 
  $$
  Then $\rho, \widetilde{\rho}$ are smooth in some neighborhoods of $\gamma, \widetilde{\gamma}$. Moreover, for all $t \in[0, a]$ and $X \in T_{\gamma(t)} M, \tilde{X} \in T_{\widetilde{\gamma}(t)} \widetilde{M}$, 
  if $|X|=|\widetilde{X}|,\langle X, \dot{\gamma}(t)\rangle=\langle\widetilde{X}, \dot{\widetilde{\gamma}}(t)\rangle$, then we have 
  $$
  \nabla^2 \widetilde{\rho}(\widetilde{X}, \widetilde{X}) \leq \nabla^2 \rho(X, X).
  $$
  We simply denote the above by $\nabla^2 \widetilde{\rho} \prec \nabla^2 \rho$ along $\gamma, \widetilde{\gamma}$.  
\end{theorem}
\begin{proof}[Proof]
  $X$ can be uniquely decomposed as follows:
  $$
  X = \langle X, \gamma'(t)\rangle {\gamma}'(t) + Y,
  $$
  where $\langle Y, \gamma'(t)\rangle = 0$. Due to the following:
  $$
  \begin{aligned}
  \nabla^2 \rho(\gamma'(t), \gamma'(t)) &= \dot{\gamma}(t) \dot{\gamma}(t) \rho - \left(\nabla_{\gamma'(t)} \gamma'(t)\right) \rho = 0, \\
  \nabla^2 \rho(\gamma'(t), Y) &= \gamma'(t) Y \rho - \left(\nabla_{\gamma'(t)} Y\right) \rho \\
  &= \gamma'(t) \left\langle Y, \frac{\partial}{\partial \rho}\right\rangle - \left\langle\nabla_{\gamma'(t)} Y, \frac{\partial}{\partial \rho}\right\rangle \\
  &= \left\langle Y, \nabla_{\gamma'(t)} \frac{\partial}{\partial \rho}\right\rangle = 0,
  \end{aligned}
  $$
  we have:
  $$
  \begin{aligned}
  \nabla^2 \rho(X, X) 
  &= \langle X, \gamma'(t)\rangle^2 \nabla^2 \rho(\gamma'(t), \gamma'(t)) \\
  &\quad + 2\langle X, \gamma'(t)\rangle \nabla^2 \rho(\gamma'(t), Y) + \nabla^2 \rho(Y, Y)  \\
  &= \nabla^2 \rho(Y, Y).
  \end{aligned}
  $$
  Similarly, we have:
  $$
  \tilde{X} = \langle\tilde{X}, \dot{\widetilde{\gamma}}(t)\rangle \dot{\widetilde{\gamma}}(t) + \widetilde{Y}, \quad \text{ and }\quad
  \nabla^2 \widetilde{\rho}(\tilde{X}, \widetilde{X}) = \nabla^2 \widetilde{\rho}(\tilde{Y}, \widetilde{Y}).
  $$
  Therefore, to prove $\nabla^2 \widetilde{\rho}(\tilde{X}, \widetilde{X}) \leq \nabla^2 \rho(X, X),$ it is equivalent to proving:
  $$
  \nabla^2 \widetilde{\rho}(\tilde{Y}, \tilde{Y}) \leq \nabla^2 \rho(Y, Y).
  $$

  At this point:
  $$
  \begin{aligned}
  & \langle Y, \gamma'(t)\rangle = 0 = \langle\widetilde{Y}, \tilde{\gamma}'(t)\rangle, \\
  &\begin{aligned}
    |Y|^2 
    &= |X-\langle X, \gamma'(t)\rangle \gamma'(t)|^2 \\
    &= |X|^2-\langle X, \gamma'(t)\rangle^2 \\
    &= |\widetilde{X}|^2-\langle\widetilde{X}, \widetilde{\gamma}'(t)\rangle^2=|\widetilde{Y}|^2 .
    \end{aligned}
  \end{aligned}
  $$
  Since $\gamma$ and $\widetilde{\gamma}$ are stably shortest geodesics, there are no conjugate points. 
  We can separately construct normal Jacobi fields $J$ and $\widetilde{J}$ along $\gamma$ and $\widetilde{\gamma}$, such that:
  $$
  J(t) = Y, \quad \widetilde{J}(t) = \widetilde{Y}.
  $$
  By Lemma 1, we only need to prove 
  \begin{equation}
    \langle \tilde{J}'\left(a\right),\tilde{J}\left(a\right)\rangle \leq \langle J'\left(a\right),J\left(a\right)\rangle. 
  \end{equation}
  In the proof of Rauch Comparison Theorem (general case), replace the definitions 
  $$
  W_{t_1}(t) = \frac{J(t)}{\left|J(t_1)\right|} \quad \text{and} \quad \tilde{W}_{t_1}(t) = \frac{\tilde{J}(t)}{\left|\tilde{J}(t_1)\right|}.
  $$
  by
  $$
  W_{t_1}(t) = J(t) \quad \text{and} \quad \tilde{W}_{t_1}(t) = \tilde{J}(t).
  $$
  Then followed the same procedure, one can prove inequality (1).
\end{proof}

Now we give some corollaries which are under the same prerequisites as Theorem 2.
\begin{corollary}
  If $f:[0, +\infty) \rightarrow \mathbf{R}$ is a $C^\infty$ increasing function (i.e., $f^\prime \geq 0$), then along $\gamma$ and $\widetilde{\gamma}$, we have:
  $$
  \nabla^2 f(\widetilde{\rho}) \prec  \nabla^2 f(\rho).
  $$  
\end{corollary}
\begin{proof}[Proof]
  Since
  $$
  \begin{aligned}
  \nabla^2 f(\rho)(X, X) & = X X f(\rho) - \left(\nabla_X X\right) f(\rho) \\
  & = f^\prime(\rho) \left(X X \rho - \left(\nabla_X X\right) \rho\right) + f^{\prime \prime}(\rho)(X \rho)^2 \\
  & = f^\prime(\rho) \nabla^2 \rho(X, X) + f^{\prime \prime}(\rho)\langle X, \dot{\gamma}(t)\rangle^2,
  \end{aligned}
  $$
  the conclusion is evident.
\end{proof}

Since $\Delta$ represents the trace of $\nabla^2$, we can also compare the Laplacian of distance functions.
\begin{corollary}
  For any $t \in (0, a]$, we have 
  $$
  \Delta \widetilde{\rho}(\widetilde{\gamma}(t)) \leq \Delta \rho(\gamma(t)).
  $$
\end{corollary}

\subsection*{Toponogov's Theorem}

In this section we state Toponogov's Theorem$^{\cite{Cheeger2008}}$, which is a powerful global generalization of the Rauch Theorem. There will be two equivalent statements, and it will be convenient to prove them simultaneously.
All indices below are to be taken modulo 3. 
\begin{definition}
  A geodesic triangle in the Riemannian manifold $\mathrm{M}$ is a set of three geodesic segments parameterized by arc length $\left(\gamma_1, \gamma_2, \gamma_3\right)$ of lengths $l_1, l_2, l_3$ such that $\gamma_i\left(l_i\right)=\gamma_{i+1}(0)$ and $l_i+l_{i+1} \geq l_{i+2}$. Set
  $$
  \alpha_i=\angle\left(-\gamma_{i+1}^{\prime}\left(l_{i+1}\right), \gamma_{i+2}^{\prime}(0)\right),
  $$
  the angle between $-\gamma_{i+1}^{\prime}\left(l_{i+1}\right)$ and $\gamma_{i+2}^{\prime}(0), 0 \leq \alpha_i \leq \pi$.  
\end{definition}
We shall specify a geodesic triangle by giving its sides $\left(\gamma_1, \gamma_2, \gamma_3\right)$.

Let $M$ be a complete manifold with $K_M \geq H$, i.e. for all plane sections $\sigma\in M$, $K(\sigma)\geq H$.
\begin{theorem}[Toponogov's Theorem A]
  Let $\left(\gamma_1, \gamma_2, \gamma_3\right)$ determine a geodesic triangle in M. Suppose $\gamma_1, \gamma_3$ are minimal and if $H>0$, suppose $L\left[\gamma_2\right] \leq \frac{\pi}{\sqrt{H}}$. 
  Then in $M^H$, the simply connected 2-dimensional space of constant curvature $H$, there exists a geodesic triangle $\left(\bar{\gamma}_1, \bar{\gamma}_2, \bar{\gamma}_3\right)$ 
  such that $L\left[\gamma_i\right]=L\left[\bar{\gamma}_i\right]$ and $\bar{\alpha}_1 \leq \alpha_1, \bar{\alpha}_3 \leq \alpha_3$. 
  Except in case $H>0$ and $L\left[\gamma_i\right]=\frac{\pi}{\sqrt{H}}$ for some $i$, the triangle in $M^H$ is uniquely determined.
\end{theorem}
\begin{theorem}[Toponogov's Theorem B]
  Let $\gamma_1, \gamma_2$ be geodesic segments in $M$ such that $\gamma_1\left(l_1\right)=\gamma_2(0)$ and $\angle\left(-\gamma_1^{\prime}\left(l_1\right), \gamma_2^{\prime}(0)\right)=\alpha$. 
  We call such a configuration a hinge $l$ and denote it by $\left(\gamma_1, \gamma_2, \alpha\right)$. Let $\gamma_1$ be minimal, and if $H>0$,
  $$
  L\left[\gamma_2\right] \leq \frac{\pi}{\sqrt{H}}
  $$
  Let $\bar{\gamma}_1, \bar{\gamma}_2 \subset M^H$ be such that $\gamma_1\left(l_1\right)=\gamma_2(0), L\left[\gamma_i\right]=L\left[\bar{\gamma}_i\right]=l_i$ and 
  $\angle\left(-\bar{\gamma}_1^{\prime}\left(l_1\right), \bar{\gamma}_2^{\prime}(0)\right)=\alpha$. Then
  $$
  \rho\left(\gamma_1(0), \gamma_2\left(l_2\right)\right) \leq \rho\left(\bar{\gamma}_1(0), \bar{\gamma}_2\left(l_2\right)\right)
  $$    
\end{theorem}
For the proof is too long but the necessary tools to prove this theorem are all given by us so far, we shall not present the proof here. 




\bibliographystyle{unsrt}
\bibliography{NoteRef}

\end{document}