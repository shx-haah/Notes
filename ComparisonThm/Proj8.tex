%!TEX program = Xelatex
\documentclass{ctexart}

\usepackage{amsmath, amssymb, amsthm, titlesec}
\usepackage[skins, breakable, theorems]{tcolorbox}
\usepackage{xparse, ocgx2, hyperref, endnotes}
\usepackage[top=2.54cm, bottom=2.54cm, left=3.18cm, right=3.18cm]{geometry}

\def\Id{\,\mathrm{d}} % 积分中的正体d
\newcommand{\norm}[1]{\left|#1\right|} % 范数

% [number within=section/...]{}{<display name>}{<style>}{<label>(cite as "Thm:...")}
\newtcbtheorem[]{tcbdefinition}{Definition}{fonttitle = \bfseries}{Def}
\newtcbtheorem[]{tcbtheorem}{Theorem}{fonttitle = \bfseries}{Thm}
\newtcbtheorem[]{tcbproposition}{Proposition}{fonttitle = \bfseries}{Pro}
\newtcbtheorem[]{tcblemma}{Lemma}{fonttitle = \bfseries}{Lem}
\newtcbtheorem[]{tcbcorollary}{Corollary}{fonttitle = \bfseries}{Cor}
\NewDocumentEnvironment{definition}{ O{} O{} } % two optional arguments
  {\tcbdefinition{#1}{#2}}
  {\endtcbdefinition}
\NewDocumentEnvironment{theorem}{ O{} O{} }
  {\tcbtheorem{#1}{#2}}
  {\endtcbtheorem}
\NewDocumentEnvironment{proposition}{ O{} O{} }
  {\tcbproposition{#1}{#2}}
  {\endtcbproposition}
\NewDocumentEnvironment{lemma}{ O{} O{} }
  {\tcblemma{#1}{#2}}
  {\endtcblemma}
\NewDocumentEnvironment{corollary}{ O{} O{} }
  {\tcbcorollary{#1}{#2}}
  {\endtcbcorollary}
\makeatletter
\newcommand\tcb@cnt@tcbdefinitionautorefname{Definition}
\newcommand\tcb@cnt@tcbtheoremautorefname{Theorem}
\newcommand\tcb@cnt@tcbpropositionautorefname{Proposition}
\newcommand\tcb@cnt@tcblemmaautorefname{Lemma}
\newcommand\tcb@cnt@tcbcorollaryautorefname{Corollary} 
\makeatother

% Sketch of proof & Remark
\newtcolorbox{sop}{
    blanker, breakable, left = 5mm,
    before skip = 10pt, after skip = 10pt,
    borderline west = {1mm}{0pt}{red},
    coltitle = black, title = \emph{Sketch of Proof}
}
\newtcolorbox{remark}{
    colback = white, fonttitle = \bfseries,
    colbacktitle = white, coltitle = black, enhanced,
    boxed title style = {sharp corners},
    attach boxed title to top left={yshift = -2mm, xshift = 5mm},
    title = Remark
}

\newcommand{\linktoprf}[1]{\hyperlink{#1}{To complete proof.}}
\newcommand{\targetprf}[1]{\hypertarget{#1}{Proof of \autoref{#1}}}

\title{Note 8: Comparison Theorems in Riemannian Geometry and Applications}
\date{} % 显示日期

\renewcommand\refname{Reference}

%%%%%%%%%%%%%%%%%%%%%%%%%%%%%%%%%%%%%%%%%%%%%%%%%%%%%
\begin{document}

\maketitle

In this Note, we first introduce some applications of Rauch Comparison Theorem. Then we will give another two Comparison Theorems, Hessian Comparison Theorem and Toponogov Comparison Theorem, 
and also some applications of these two theorems. 


\subsection*{Applications of Rauch Comparison Theorem}

Recall the Rauch Comparison Theorem in Note 7. By Hopf-Rinow Theorem, if $M$ is a complete Riemannian manifold with non-positive sectional curvature, $\exp_p$ is defined on the whole $T_pM$. 
Then for any $p \in M$, $\exp_{p}: T_pM \rightarrow M$ is a distance-expanding map. 
Furthermore, there is an interesting corollary, which can be found in \cite{WuHongXi2014}, Corollary 3, p.119.
\begin{corollary}
    Let $M$ be a simply connected complete Riemannian manifold with non-positive sectional curvature, and let $ABC$ be a geodesic triangle in $M$ 
    (i.e. all sides of the triangle are minimal geodesics), with corresponding angles $A, B, C$, and corresponding side $a, a, c$, then 

    (1) $a^2+b^2-2 a b \cos C \leq c^2$, 

    (2) $A+B+C \leq \pi$. 
\end{corollary}
\begin{proof}[Proof]
    Let $x$ be the vertex of angle $C$, and refer to Figure 8.2 in the tangent space $T_x M$. Draw radial line segments $\overline{O p}$ and $\overline{O q}$ with lengths $a$ and $b$, respectively, 
    such that $\exp_x$ maps $\overline{O p}$ and $\overline{O q}$ to the sides of the geodesic triangle with lengths $a$ and $b$. Let $\zeta$ be the $\exp_x^{-1}$ image of side $c$. 
    Since $\exp_x$ is a distance-expanding map, we have $L(\zeta) \leq c$, so the length of the line segment $\overline{p q}$ in $T_x M$ is $L(\overline{p q}) \leq L(\zeta) \leq c$.
    The law of cosines for the triangle $O p q$ in $T_x M$ is given by:
    $$
    a^2 + b^2 - 2ab\cos C = L(\overline{p q})^2,
    $$
    which proves (1).
    In $\mathbb{R}^2$, construct a triangle with boundary lengths $a$, $b$, and $c$. This is possible because the sides of a geodesic triangle are always the shortest geodesic paths, 
    and the sum of the lengths of any two sides must be greater than the length of the third side. Let $A', B', C'$ denote the angles of this triangle in $\mathbb{R}^2$.
    From (1), we know that $C \leq C'$. Similarly, we can deduce $A \leq A'$ and $B \leq B'$. Thus, (2) is proven.
\end{proof}
This corollary is a special case of Toponogov Comparison Theorem. The proof of the Toponogov theorem essentially relies on the Rauch comparison theorem,
and the method is similar to the proof of this corollary. However, it typically involves conjugate points, making the situation more complex.

\begin{theorem}
  Let $M$ and $\tilde{M}$ be $m$-dimensional Riemannian manifolds, and let $K$ and $\tilde{K}$ be the sectional curvatures, respectively. Consider a point $p$ in $M$ and $\tilde{p}$ in $\tilde{M}$. 
  Let $\varphi: T_p M \rightarrow T_{\tilde{p}} \tilde{M}$ be an isometric linear isomorphism. Assume that $W$ and $\tilde{W}=\varphi(W)$ are neighborhoods of the origin in $T_p M$ and $T_{\tilde{p}} \tilde{M}$, 
  respectively, such that the exponential mappings
  $$
  \exp_p: W \rightarrow \exp_p(W) \subset M \quad \text{ and } \quad \exp_{\tilde{p}}: \tilde{W} \rightarrow \exp_{\tilde{p}}(\tilde{W}) \subset \tilde{M}
  $$
  are both diffeomorphisms. If for any corresponding points
  $$
  q \in \exp_p(W) \quad \text{ and } \quad \tilde{q}=\exp_{\tilde{p}} \circ \varphi \circ (\exp_q)^{-1}(q) \in \exp_{\tilde{p}}(\tilde{W}),
  $$
  and for any two-dimensional sections $\sigma \subset T_q M$ and $\tilde{\sigma} \subset T_{\tilde{q}} \tilde{M}$, the following inequality holds:
  $$
  K(\sigma) \geq \tilde{K}(\tilde{\sigma}),
  $$
  then, for any smooth curve $\beta(u)$ in $W$ with $0 \leq u \leq 1$, the following inequality holds:
  $$
  L(\exp_p \circ \beta) \leq L(\exp_{\tilde{p}} \circ (\varphi \circ \beta)).
  $$
\end{theorem}
\begin{proof}[Proof]
  Without loss of generality, we can assume that $\beta$ is nonzero everywhere (i.e., $\beta$ does not pass through the origin).
  Consider the geodesic variation:
  $$
  \alpha(t, u) = \exp_p(t \beta(u)), \quad 0 \leq t \leq 1, \quad 0 \leq u \leq 1.
  $$
  Let
  $$
  V(t, u) = \alpha_{*(t, u)}\left(\frac{\partial}{\partial u}\right) = \left(\exp_p\right)_{* t \beta(u)}\left(t \beta^{\prime}(u)\right).
  $$
  Then:
  $$
  \begin{aligned}
  V(1, u)  &= \left(\exp_p\right)_{* \beta(u)}\left(\beta^{\prime}(u)\right), \\
  L\left(\exp_p \circ \beta\right) &= \int_0^1|V(1, u)| \mathrm{d} u.
  \end{aligned}
  $$
  Similarly, we have:
  $$
  L\left(\exp_{\tilde{p}} \circ \varphi \circ \beta\right) = \int_0^1|\tilde{V}(1, u)| \mathrm{d} u,
  $$
  where:
  $$
  \tilde{V}(1, u) = \frac{\mathrm{d}}{\mathrm{d} u}\left(\exp_{\tilde{p}}(\varphi \circ \beta(u))\right) = \left(\exp_{\tilde{p}}\right)_{* \varphi \circ \beta(u)}\left(\varphi\left(\beta^{\prime}(u)\right)\right).
  $$

  For any fixed $u \in (0,1)$, consider the vector fields $V(t, u)$ and $\tilde{V}(t, u)$ along the geodesics $\gamma_u$ and $\tilde{\gamma}_u$, respectively, where
  $$
  \gamma_u(t) = \exp_p(t \beta(u)) \quad \text{ and } \quad \tilde{\gamma}_u(t) = \exp_{\tilde{p}}(t \varphi \circ \beta(u)). 
  $$
  According to the process of defining Jacobi fields, $V(t, u)$ and $\tilde{V}(t, u)$ are Jacobi fields along $\gamma_u$ and $\tilde{\gamma}_u$, respectively. By definition:
  $$
  \begin{aligned}
  & V(0, u) = \tilde{V}(0, u) = 0, \\
  & \frac{\mathrm{D} V}{\mathrm{d} t}(0, u) = \beta^{\prime}(u), \quad \frac{\mathrm{D} \tilde{V}}{\mathrm{d} t}(0, u) = \varphi\left(\beta^{\prime}(u)\right), \\
  & \gamma_u^{\prime}(0) = \beta(u), \quad \tilde{\gamma}_u^{\prime}(0) = \varphi(\beta(u)).
  \end{aligned}
  $$
  Based on the assumption that, for any $X \in T_{\gamma_u(t)} M$ and $\tilde{X} \in T_{\tilde{\gamma}_u(t)} \tilde{M}$:
  $$
  K\left(\gamma_u^{\prime}, X\right) \geq \tilde{K}\left(\tilde{\gamma}_u^{\prime}, \tilde{X}\right),
  $$
  we can apply Rauch Comparison Theorem, which implies:
  $$
  |V(t, u)| \leq |\tilde{V}(t, u)|.
  $$
  In particular:
  $$
  |V(1, u)| \leq |\tilde{V}(1, u)|,
  $$
  and thus:
  $$
  L\left(\exp_p \circ \beta\right) \leq L\left(\exp_{\tilde{p}} \circ \varphi \circ \beta\right).
  $$
\end{proof}




\bibliographystyle{unsrt}
\bibliography{NoteRef}

\end{document}